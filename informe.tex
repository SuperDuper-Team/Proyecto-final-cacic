\documentclass[11pt,a4paper]{article}
\usepackage[utf8]{inputenc}
\usepackage[spanish,es-tabla]{babel}
\usepackage{amsmath}
\usepackage{amsfonts}
\usepackage{amssymb}
\usepackage{graphicx}
\usepackage{float}
\usepackage{cite}

\usepackage{vmargin}

\setpapersize{A4}
\setmargins{2.5cm}              % margen izquierdo
{1.5cm}                         % margen superior
{16.5cm}                        % anchura del texto
{23.42cm}                       % altura del texto
{10pt}                          % altura de los encabezados
{1cm}                           % espacio entre el texto y los encabezados
{0pt}                           % altura del pie de página
{2cm}                           % espacio entre el texto y el pie de página

\title{ 
    Emisión de titulos universitarios
    en la Blockchain
}
\author{
    Saez, Lautaro Andres \\ \small{ LautaroAndresSaez@gmail.com } 
    \and 
    Riperto, Adriel Aaron \\ \small{ aaron.ariperto@gmail.com } 
}
\date{\today}

\begin{document}
    \maketitle

    \section{Introducción}

    \section{Estado del arte}

        Esta sección tiene como objetivo tratara el estado actual de la tematica a abordar.
        Debido a que blockchain se encuentra en constante crecimiento, no hay una gran cantidad 
        de trabajos implementados en el ambito sobre los titulos academicos. Pero se investigo 
        y encontraron dos propuestas sobre el tema. %Correguir implementaciones!

        \subsection{Blockchain federal Argentina (BFA)}


        Blockchain Federal Argentina es una plataforma multiservicios abierta y participativa pensada en integrar servicios y aplicaciones
        sobre blockchain \cite{Blockchain-federal-Argentina}. % Referencia https://bfa.ar/bfa/que-es-bfa
        
        BFA propone multiples casos de uso para blockchain, entre ellos se encuentra una propuesta donde el alumno pueda solicitar
        el titulo academico, la universidad valide la aprobacion de las materias y el ministerio de educacion certifique el titulo.

        Esto se propone lograrlo empleando “sellos de tiempo” donde es posible garantizar que los documentos involucrados 
        no puedan ser alterados. Estos generan digestos criptográficos (hash) de historias académicas o títulos que quedan 
        almacenados en la blockchain y a través de ellos se puede garantizar que los mismos no han recibido modificaciones 
        indebidas en todo el proceso \cite{titulos-academicos}. % Referencia https://bfa.ar/blockchain/casos-de-uso/titulos-academicos
 
        Las ventajas que presenta esta solucion son:
        
        \begin{itemize}
            \item Garantiza que no sea posible alterar la 
            información de las actas sin que esa modificación sea detectada y así aumenta la confianza en la autenticidad de los títulos emitidos. 
            \item Transparencia en el proceso de digitalización.
            \item Permite un contexto de confianza entre organismos y partes interesadas. 
            \item Auditable.
            \item Permite demostrar que no existen actos de negligencia (títulos truchos) en torno a la emisión de títulos.
            \item Se podría dar diferentes vigencias a certificado de títulos en trámite, o similares, certificadas en la blockchain.
        \end{itemize}

        Por otro lado se proponen posibles mejoras a esta implementación creando un portafolios digital, lo cual 
        permitiria modificar los permisos de acceso.

        \begin{figure}
            \centering
            \includegraphics[width=\textwidth]{Img/cuadro_problematica.png}
            \caption{}
            \label{fig:cuadro_problematica}
        \end{figure}

        \subsection{Smart Degress}
        
        Smart Degress es una plataforma que permite al usuario mediante una Dapp la registracion y
        certificacion de diplomas y titulos academicos. Esto lo hace basandose en la premisa de que el 
        exito en el mercado laboral, depende de la certificacion de los titulos obtenidos.

        Para lograr su objetivo emplean la tecnologia blockchain con la cual proporcionan y añade una mayor agilidad,
        comodidad y seguridad. Con lo cual permite gestionar y compartir el titulo universitario en plataformas de trabajo, 
        reclutadores y terceros.


    \section{Objetivos}

        \subsection{Objetivo principal}

        Implementar las bases de un sistema en la blockchain que permita la generacion 
        y autenticacion de los titulos universitarios. 
        
        \subsection{Objetivos secundarios}

        \begin{itemize}
            \item Analizar los procesos que implican la generacion y validacion de un titulo.
            \item Evaluar el ecosistema de tecnologias a implementar.
            \item Diseñar los componentes de la aplicaciones.
            \item Implementar el sistema utilizando la BFA. 
        \end{itemize}


    \section{Desafíos}

        Esta seccion tiene como objetivo describir los desafios que se deben afrontar para poder realizar la aplicacion propuesta.

        \subsection{Modelo y procesos del negocio} 

            El primer desafio planteado implica analizar y comprender por parte del equipo, los procesos que implican la 
            generacion y validacion de un titulo universitario. Esto tiene como objetivo replicar y mejorar los procesos 
            en el sistema a construirse.
        
        \subsection{Aprendizaje de tecnologias} 

            El segundo desafio sera aprender las siguientes tecnologias:
            
            \begin{itemize}
                \item Blockchain Federal Argentina
                \item React
                \item NodeJS
            \end{itemize}
            
            Las tecnologias planteadas tienen como objetivo crear una arquitectura viable y limpia para el desarrollo del sistema.
            
            
        \subsection{Difusion del sistema} 

            Por ultimo se debe realizar una difusion del ecosistema planteado y lograr una aceptacion por parte de los alumnos y las diferentes entidades.
            El objetivo de este desafio es lograr una mejora en la gestion de titulos, eliminando los intermediarios y facilitando la accesibilidad 
            a los documentos universitarios. 

    \section{Etapas}

        La forma de trabajo a llevar a cabo sera la siguiente:

        \begin{table}[H]
            \centering
            \begin{tabular}{|c|c|}
                \hline Activdad & Tiempo estimado \\ 
                \hline Prueba de concepto en la BFA & \\
                \hline Planificacion de los smart contracts & \\
                \hline Analisis & \\
                \hline Diseño de la arquitectura & \\ 
                \hline Desarrollo de los smart contracts & \\
                \hline Desarrollo del backend & \\ 
                \hline Desarrollo del frontend & \\ 
                \hline Despliegue sobre BFA & \\
                \hline Tiempo total & \\
                \hline
            \end{tabular}
            \label{tab:etapas}
        \end{table}

 
    \section{Conclusiones}


\bibliographystyle{IEEEannot}
\bibliography{biblio}
\end{document}